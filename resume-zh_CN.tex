% !TEX TS-program = xelatex
% !TEX encoding = UTF-8 Unicode
% !Mode:: "TeX:UTF-8"

\documentclass{resume}
\usepackage{zh_CN-Adobefonts_external} % Simplified Chinese Support using external fonts (./fonts/zh_CN-Adobe/)
%\usepackage{zh_CN-Adobefonts_internal} % Simplified Chinese Support using system fonts
\usepackage{linespacing_fix} % disable extra space before next section
\usepackage{cite}

\begin{document}
\pagenumbering{gobble} % suppress displaying page number

\name{宋彤雨}

% {E-mail}{mobilephone}{homepage}
% be careful of _ in emaill address
\contactInfo{(+86) 180-8093-2513}{SDNanyflow@gmail.com}{}{}
% {E-mail}{mobilephone}
% keep the last empty braces!
%\contactInfo{xxx@yuanbin.me}{(+86) 131-221-87xxx}{}
 
% \section{个人总结}
% 本人在校成绩优秀、乐观向上,工作负责、自我驱动力强、热爱尝试新事物,认同开放、连接、共享的Web在未来的不可替代性。在校期间长期从事可视分析(Web的2D/3D时空可视化)相关研究,对Web技术发展趋势及前端工程化解决方案有浓厚兴趣。\textbf{现任职于阿里巴巴集团。}

% \section{\faGraduationCap\ 教育背景}
\section{教育背景}
\datedsubsection{\textbf{电子科技大学},通信与信息系统,\textit{在读博士研究生}}{2014.9 - 2022.6}
% \ \textbf{排名11/133(前10\%)},中国科学院大学学业奖学金(2次),IEEE Student member,预计2018年6月毕业
\datedsubsection{\textbf{电子科技大学},网络工程,\textit{工学学士}}{2010.9 - 2014.6}
% \ \textbf{排名5/102(前5\%)},国家励志奖学金,人民奖学金(7次),科技竞赛奖(2次),北京市普通高等学校优秀毕业生,北京理工大学优秀毕业生,软件学院金牌毕业生,优秀团员/优秀学生(5次)
\ \textbf{排名5/102},国家奖学金,人民奖学金(3次)
% \datedsubsection{\textbf{荷兰 莱顿大学},计算机科学与技术,\textit{国家留学基金委公派交换生}}{2015.3 - 2015.5}
% \ 2014年中国政府奖学金(\textit{http://www.csc.edu.cn/}),DID-ACTE项目交换生(\textit{http://did-acte.org/})
\section{学术成果}
\datedsubsection{\textbf{学术论文}}{}
\begin{enumerate}
  \item \textbf{Tongyu Song}, X. Tan, J. Ren, et al., DRAM: A DRL-based Resource Allocation Scheme for MAR in MEC.
  Digital Communications and Networks(DCN), 2022.(SCI期刊,中科院一区)
  \\https://doi.org/10.1016/j.dcan.2022.04.014
  \item \textbf{Tongyu Song}, W. Hu, X. Tan, et al., FAST-RAM: a Fast AI-assistant Solution for Task Offloading and Resource Allocation in MEC. IEEE Global Communications Conference (GLOBECOM), 2020.(EI会议,CCF C类推荐会议)
  % \\http://ieeexplore.ieee.org/stamp/stamp.jsp?tp=\&arnumber=9322645\&isnumber=9321973
  \item \textbf{Tongyu Song}, W. Hu, X. Tan, et al., ARM: An Accelerator for Resource Allocation in Mobile Edge Computing. IEEE Global Communications Conference (GLOBECOM), 2019.(EI会议,CCF C类推荐会议)
  % \\http://ieeexplore.ieee.org/stamp/stamp.jsp?tp=\&arnumber=9014042\&isnumber=9013108
  \item \textbf{Tongyu Song}, W. Hu, W. Xu, et al., FAIR-AREA: A Fast AI-Based Joint Optimization of Rate Adaptation and Resource Allocation for DASH. IEEE Global Communications Conference (GLOBECOM), 2019.(EI会议,CCF C类推荐会议)
  % \\http://ieeexplore.ieee.org/stamp/stamp.jsp?tp=\&arnumber=9013477\&isnumber=9013108
  \item \textbf{Tongyu Song}, S. Wang, J. Ren, et al., JRA2: Joint Optimization of Resource Allocation and Rate Adaptation for DASH Services. IEEE International Conference on Communications (ICC), 2018.(EI会议,CCF C类推荐会议)
  \\http://ieeexplore.ieee.org/stamp/stamp.jsp?tp=\&arnumber=8422841\&isnumber=8422068
  \item \textbf{Tongyu Song}, X. Tan, J. Ren, et al., COLLAR: Adaptation Scheme for Mobile Augmented Reality in Mobile Edge Computing. Future Generation Computer Systems.(SCI期刊,中科院一区,评审中)
  \item\textbf{Tongyu Song}, X. Guo, J. Ren, et al., CO-CAST: Scheme for routing and bandwidth allocation for multiple consortium blockchain based on deep reinforcement learning. Future Generation Computer Systems.(SCI期刊,中科院一区,评审中)
  \item\textbf{Tongyu Song}, J. Zheng, J. Ren, et al., CO-NEXT: Scheme for intelligent topology optimization for ad-hoc network based on multi-agent DRL. IEEE Internet of Things Journal.(SCI期刊,中科院一区,评审中)
\end{enumerate}
\datedsubsection{\textbf{发明专利}}{}
\begin{enumerate}
    \item 一种多联盟链共识算法的网络时延优化方法, CN202110591340.6.(第一发明人)
    \item 基于深度学习的智能视频码率调整及带宽分配方法, CN202110097764.7.(第一发明人)
    \item 基于智能体深度增强学习的多边缘基站联合缓存替换方法, CN202110599821.1.(第一发明人)
    \item 基于多智能体增强学习的WSN能量效率优化路由方法, CN202210378218.5.(第二发明人,审批中)
\end{enumerate}
\datedsubsection{\textbf{学术专著}}{}
《AI技术应用:网络资源管理》,人民邮电出版社(撰写中,2022.10出版)
% \section{\faCogs\ IT 技能}
% \section{技术能力} % increase linespacing [parsep=0.5ex]
% \begin{itemize}[parsep=0.2ex]
%   \item \textbf{编程语言}: JavaScript (ECMAScript, Node.js), HTML/CSS, Python, Go, SQL, C, Shell
%   \item \textbf{操作系统,数据库与工程构建}: Linux/macOS/MySQL/MongoDB/Git/webpack/Progressive Web App
%   \item \textbf{关键词}: React/Vue.js/D3.js(SVG)/three.js(canvas, WebGL)/chrome extension/Express
% \end{itemize}

% \end{itemize}

\section{科研项目}
\datedsubsection{国家重点基础研究发展计划(973计划),2013CB329103,资源动态适配机制与理论}{373万元,2013/01/01-2017/08/31}

\datedsubsection{国家重点研发计划,2020YFB1807804,随愿共享的业务能力协同互联技术}{540万元,2020/12/01-2023/11/30}

\datedsubsection{国家重点研发计划,2020YFB1807805,按需服务的智联网管控原型系统研制}{73.5万元,2020/12/01-2023/11/30}

国家重点研发计划,2019YFB1802802,差异化服务智能适配与路由,162万元,2020/01/01-2022/12/31

国家自然科学基金,面上项目,6167011865,软件定义网络中应用与网络合作的资源分配模型及机制研究,80.5440万元,2017/01/01-2020/12/31

国家自然科学基金,面上项目,62072079,面向下一代网络的可编程测量架构及关键测量方法,58万元国家自然科学基金联合基金项目,U20A20156,类生物免疫机制的网络安全防护理论与方法,269万元,2021/01/01-2024/12/31

国家自然科学基金青年基金,62001087,基于多智能体协同学习的网络资源管控机制研究,24万元,2021/01/01-2023/12/31

中央军委装备发展部,“十三五”全军公用信息系统装备预先研究项目, JZX6Y202 001010034,网络流量智能规划技术,2020-01至2021-06,226万,在 研,参与

四川省科技计划项目科技支撑计划,2016GZ0138,云数据中心网络能量感知的关键技术及其应用,20万元,2016/01/01-2018/12/31

四川省市场监督局后补助国际标准制定《信息中心网络部署架构》等2项,60万元,2020/12/1-2024/12/31

四川省市场监督局后补助国际标准制定《信息中心网络中名字解析服务的要求》,50万元,2019/08/-2023.12
\datedsubsection{\textbf{阿里巴巴集团 | Alibaba}, 前端开发工程师}{2017.6-2017.9}
\begin{itemize}
%   \item 飞猪北京前端团队全面负责各交通线的票务(机票/火车票/汽车票) web 应用与事业群基础架构研发
  \item 独立负责车站地图开发(React),通过HTML5 本地存储及JSBridge实现在阿里全系应用中发布上线
  \item 独立负责BU SPM chrome插件开发,支付成功/订单详情等页面的开发与交叉营销的接入工作
\end{itemize}



\section{竞赛获奖}
% increase linespacing [parsep=0.5ex]
\begin{itemize}[parsep=0.2ex]
  \item 第一届全国高校软件定义网络(SDN)应用创新开发大赛\textbf{全国一等奖}2014.8\\ \textit{https://www2.scut.edu.cn/sdn2014/2014/0904/c4242a75928/page.htm}
\end{itemize}

% \section{\faHeartO\ 项目/作品摘要}
% \section{项目/作品摘要}
% \datedline{\textit{An Integrated Version of Security Monitor Vis System}, https://hijiangtao.github.io/ss-vis-component/ }{}
% \datedline{\textit{Dark-Tech}, https://github.com/hijiangtao/dark-tech/ }{}
% \datedline{\textit{融合社交网络数据挖掘的电视节目可视化分析系统}, https://hijiangtao.github.io/variety-show-hot-spot-vis/}{}
% \datedline{\textit{LeetCodeOJ Solutions}, https://github.com/hijiangtao/LeetCodeOJ}{}
% \datedline{\textit{Info-Vis}, https://github.com/ISCAS-VIS/infovis-ucas}{}


% \section{\faInfo\ 社会实践/其他}
\section{其它实践}
% increase linespacing [parsep=0.5ex]
\begin{itemize}[parsep=0.2ex]
  \item 乐于参与开源社区讨论,\textbf{参与翻译 Vue.js, webpack, WebAssembly, Babel 文档,印记中文成员}
  \item 中国科学院大学2016秋季学期可视化与可视分析课程助教,\textit{http://vis.ios.ac.cn/infovis-ucas/}
  \item 未来论坛学生会成员、北理社联新闻信息中心主任、北理工软件学院学生会宣传部副部长(2012-2016)
  \item 2013-2015 北京市共青团“温暖衣冬”志愿者,第九届园博会志愿者,2014 FLL机器人世锦赛志愿者
\end{itemize}

%% Reference
%\newpage
%\bibliographystyle{IEEETran}
%\bibliography{mycite}
\end{document}
